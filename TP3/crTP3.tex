\documentclass[a4paper,11pt]{exam}
\usepackage{style/styleCompteRendu}



\begin{document}

\Noms{Marie Dupond \\ Maurice Durand}
\Titre{TP3 -  Processus, droits et communication}


%/////////////////////////////////////////////////////////////////////////////////////////////////////////////////////
\exercice{Processus}
{
	Manipulation de processus.
	\begin{enumerate}
	\item Visualiser les processus en cours par la commande \texttt{ps}.
\reponse{...}		
	\item Lancer gedit en tâche de fond.
\reponse{\commande{>...}}
	\item Relancer la commande \texttt{ps}.
\reponse{.(commentaires)..}	
	\item Lancer la commande \texttt{ps -al}.
	    \begin{enumerate}
	    \item Quelles sont les PID et les PPID des processus \texttt{ps} et \texttt{gedit}?
\reponse{...}
	    \item Quels sont les pères de ces deux processus?
\reponse{...}
	    \end{enumerate}
	\item Tuer le processus \texttt{gedit} par la commande \texttt{kill}.
	\item Bloquer un processus :
	    \begin{itemize}
	    \item Lancer un processus puis taper \texttt{Ctrl-Z}. 
	    \item Passer le en tâche de fond par la commande \texttt{bg}.
\reponse{...}
	    \end{itemize}
	 \item Afficher la charge CPU des applications liées à votre nom d'utilisateur. Laquelle consomme le plus?
\reponse{\commande{>...}\ ...}	
 	\item Utiliser la commande \texttt{time} avant une commande pour déterminer le temps consommé par cette commande.
 \reponse{\commande{>...}\\Commentaires : ...}	
 	\item Utiliser la commande \texttt{>echo ``bash endormi'';sleep 15;echo ``bash reveillé''} pour endormir le processus bash.
 \reponse{\commande{>...}\\Commentaires : ...}	
	\end{enumerate}
}

%/////////////////////////////////////////////////////////////////////////////////////////////////////////////////////

\exercice{Droits}
{
	  Créer dans le répertoire \texttt{TP3} le fichier texte \texttt{scriptTP3} contenant le texte suivant : \fbox{\texttt{echo il y a \$\# paramètres : \$*}}
	  \begin{enumerate}
	  \item Visualiser les droits de ce fichier. Pouvez-vous l’exécuter?
 \reponse{\commande{>...}\\Commentaires : ...}		  
	  \item Modifier en les droits pour pouvoir l’exécuter. Qu'obtenez-vous?
 \reponse{\commande{>...}\\Commentaires : ...}		  
	  \item Modifier les droits sous les deux formes possibles.
 \reponse{\commande{>...\\>...\\>...}}		
 	  \item Où faudrait-il placer ce fichier pour pouvoir l'utiliser quelque soit le répertoire courant?
  \reponse{}
	  \end{enumerate}
}
 
%/////////////////////////////////////////////////////////////////////////////////////////////////////////////////////
\exercice{Communication}
{
	    Connectez-vous au compte d'un de vos camarades par \textbf{sftp}.
	    \begin{enumerate}
	    \item Déplacez-vous dans votre compte et le compte distant. Après chaque déplacement, utilisez les commandes \texttt{pwd} et  \texttt{lpwd}.
	    \item Lister le contenu du répertoire courant sur votre compte et le compte distant.
 \reponse{\commande{>...}}		    
	    \item Copier sur votre compte un fichier du compte distant.
 \reponse{\commande{>...}}		    
	    \item Vérifier sur votre console que ce fichier a bien été copié.
 \reponse{\commande{>...} ...}	    
	    \item Copier sur le compte distant un fichier de votre compte.
 \reponse{\commande{>...}}	
 	    \item Vérifier sur votre console que ce fichier a bien été copié.
 \reponse{\commande{>...} ...}
            \item Afficher le nombre de fichiers PDF de plus de 200ko sur le compte distant. 
 \reponse{\commande{>...} ...}
	    \end{enumerate}
}



\end{document}


