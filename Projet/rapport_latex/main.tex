%%%%%%%%%%%%%%%%%%%%%%%%%%%%%%%%%%%%%%%%%
% Lachaise Assignment
% LaTeX Template
% Version 1.0 (26/6/2018)
%
% This template originates from:
% http://www.LaTeXTemplates.com
%
% Authors:
% Marion Lachaise & François Févotte
% Vel (vel@LaTeXTemplates.com)
%
% License:
% CC BY-NC-SA 3.0 (http://creativecommons.org/licenses/by-nc-sa/3.0/)
% 
%%%%%%%%%%%%%%%%%%%%%%%%%%%%%%%%%%%%%%%%%

%----------------------------------------------------------------------------------------
%	PACKAGES AND OTHER DOCUMENT CONFIGURATIONS
%----------------------------------------------------------------------------------------

\documentclass{article}

%%%%%%%%%%%%%%%%%%%%%%%%%%%%%%%%%%%%%%%%%
% Lachaise Assignment
% Structure Specification File
% Version 1.0 (26/6/2018)
%
% This template originates from:
% http://www.LaTeXTemplates.com
%
% Authors:
% Marion Lachaise & François Févotte
% Vel (vel@LaTeXTemplates.com)
%
% License:
% CC BY-NC-SA 3.0 (http://creativecommons.org/licenses/by-nc-sa/3.0/)
% 
%%%%%%%%%%%%%%%%%%%%%%%%%%%%%%%%%%%%%%%%%

%----------------------------------------------------------------------------------------
%	PACKAGES AND OTHER DOCUMENT CONFIGURATIONS
%----------------------------------------------------------------------------------------

\usepackage{keystroke} % https://ctan.crest.fr/tex-archive/macros/latex/contrib/keystroke/key-test.pdf

\usepackage{amsmath,amsfonts,stmaryrd,amssymb} % Math packages

\usepackage{minted} 

\usepackage{enumerate} % Custom item numbers for enumerations

\usepackage[ruled]{algorithm2e} % Algorithms

\usepackage[framemethod=tikz]{mdframed} % Allows defining custom boxed/framed environments

\usepackage{listings} % File listings, with syntax highlighting
\lstset{
	basicstyle=\ttfamily, % Typeset listings in monospace font
}

\usepackage{dirtree}

\usepackage{xcolor}

\usepackage{graphicx}

\usepackage{hyperref}
%----------------------------------------------------------------------------------------
%	DOCUMENT MARGINS
%----------------------------------------------------------------------------------------

\usepackage{geometry} % Required for adjusting page dimensions and margins

\geometry{
	paper=a4paper, % Paper size, change to letterpaper for US letter size
	top=2.5cm, % Top margin
	bottom=3cm, % Bottom margin
	left=2.5cm, % Left margin
	right=2.5cm, % Right margin
	headheight=14pt, % Header height
	footskip=1.5cm, % Space from the bottom margin to the baseline of the footer
	headsep=1.2cm, % Space from the top margin to the baseline of the header
	%showframe, % Uncomment to show how the type block is set on the page
}

%----------------------------------------------------------------------------------------
%	FONTS
%----------------------------------------------------------------------------------------

\usepackage[utf8]{inputenc} % Required for inputting international characters
\usepackage[T1]{fontenc} % Output font encoding for international characters

\usepackage{XCharter} % Use the XCharter fonts

%----------------------------------------------------------------------------------------
%	COMMAND LINE ENVIRONMENT
%----------------------------------------------------------------------------------------

% Usage:
% \begin{commandline}
%	\begin{verbatim}
%		$ ls
%		
%		Applications	Desktop	...
%	\end{verbatim}
% \end{commandline}

\mdfdefinestyle{commandline}{
	leftmargin=10pt,
	rightmargin=10pt,
	innerleftmargin=15pt,
	middlelinecolor=black!50!white,
	middlelinewidth=2pt,
	frametitlerule=false,
	backgroundcolor=black!5!white,
	frametitle={Command Line},
	frametitlefont={\normalfont\sffamily\color{white}\hspace{-1em}},
	frametitlebackgroundcolor=black!50!white,
	nobreak,
}

% Define a custom environment for command-line snapshots
\newenvironment{commandline}{
	\medskip
	\begin{mdframed}[style=commandline]
}{
	\end{mdframed}
	\medskip
}

%----------------------------------------------------------------------------------------
%	FILE CONTENTS ENVIRONMENT
%----------------------------------------------------------------------------------------

% Usage:
% \begin{file}[optional filename, defaults to "File"]
%	File contents, for example, with a listings environment
% \end{file}

\mdfdefinestyle{file}{
	innertopmargin=1.6\baselineskip,
	innerbottommargin=0.8\baselineskip,
	topline=false, bottomline=false,
	leftline=false, rightline=false,
	leftmargin=2cm,
	rightmargin=2cm,
	singleextra={%
		\draw[fill=black!10!white](P)++(0,-1.2em)rectangle(P-|O);
		\node[anchor=north west]
		at(P-|O){\ttfamily\mdfilename};
		%
		\def\l{3em}
		\draw(O-|P)++(-\l,0)--++(\l,\l)--(P)--(P-|O)--(O)--cycle;
		\draw(O-|P)++(-\l,0)--++(0,\l)--++(\l,0);
	},
	nobreak,
}

% Define a custom environment for file contents
\newenvironment{file}[1][File]{ % Set the default filename to "File"
	\medskip
	\newcommand{\mdfilename}{#1}
	\begin{mdframed}[style=file]
}{
	\end{mdframed}
	\medskip
}

%----------------------------------------------------------------------------------------
%	NUMBERED QUESTIONS ENVIRONMENT
%----------------------------------------------------------------------------------------

% Usage:
% \begin{question}[optional title]
%	Question contents
% \end{question}

\mdfdefinestyle{question}{
	innertopmargin=1.2\baselineskip,
	innerbottommargin=0.8\baselineskip,
	roundcorner=5pt,
	nobreak,
	singleextra={%
		\draw(P-|O)node[xshift=1em,anchor=west,fill=white,draw,rounded corners=5pt]{%
		Question \theQuestion\questionTitle};
	},
}

\newcounter{Question} % Stores the current question number that gets iterated with each new question

% Define a custom environment for numbered questions
\newenvironment{question}[1][\unskip]{
	\bigskip
	\stepcounter{Question}
	\newcommand{\questionTitle}{~#1}
	\begin{mdframed}[style=question]
}{
	\end{mdframed}
	\medskip
}

%----------------------------------------------------------------------------------------
%	WARNING TEXT ENVIRONMENT
%----------------------------------------------------------------------------------------

% Usage:
% \begin{warn}[optional title, defaults to "Warning:"]
%	Contents
% \end{warn}

\mdfdefinestyle{warning}{
	topline=false, bottomline=false,
	leftline=false, rightline=false,
	nobreak,
	singleextra={%
		\draw(P-|O)++(-0.5em,0)node(tmp1){};
		\draw(P-|O)++(0.5em,0)node(tmp2){};
		\fill[black,rotate around={45:(P-|O)}](tmp1)rectangle(tmp2);
		\node at(P-|O){\color{white}\scriptsize\bf !};
		\draw[very thick](P-|O)++(0,-1em)--(O);%--(O-|P);
	}
}

% Define a custom environment for warning text
\newenvironment{warn}[1][Warning:]{ % Set the default warning to "Warning:"
	\medskip
	\begin{mdframed}[style=warning]
		\noindent{\textbf{#1}}
}{
	\end{mdframed}
}

%----------------------------------------------------------------------------------------
%	INFORMATION ENVIRONMENT
%----------------------------------------------------------------------------------------

% Usage:
% \begin{info}[optional title, defaults to "Info:"]
% 	contents
% 	\end{info}

\mdfdefinestyle{info}{%
	topline=false, bottomline=false,
	leftline=false, rightline=false,
	nobreak,
	singleextra={%
		\fill[black](P-|O)circle[radius=0.4em];
		\node at(P-|O){\color{white}\scriptsize\bf i};
		\draw[very thick](P-|O)++(0,-0.8em)--(O);%--(O-|P);
	}
}

% Define a custom environment for information
\newenvironment{info}[1][Info:]{ % Set the default title to "Info:"
	\medskip
	\begin{mdframed}[style=info]
		\noindent{\textbf{#1}}
}{
	\end{mdframed}
}


%----------------------------------------------------------------------------------------
%	EQUATION
%----------------------------------------------------------------------------------------

\newcommand{\equabezier}{
	\begin{displaymath}
		P(t) = P_0(1-t)^3+3P_1t(1-t)^2+3P_2t^2(1-t)+P_3t^3	
	\end{displaymath}
} % Include the file specifying the document structure and custom commands

%----------------------------------------------------------------------------------------
%	ASSIGNMENT INFORMATION
%----------------------------------------------------------------------------------------

\title{Système et réseaux : rapport} % Title of the assignment
\author{Valentin VERSTRACTE \& Evan PETIT}

\date{L3 --- \today} % University, school and/or department name(s) and a date



%----------------------------------------------------------------------------------------

\begin{document}

\maketitle % Print the title


%----------------------------------------------------------------------------------------
%	Introduction
%----------------------------------------------------------------------------------------

\section{Structure générale de l'application}
La structure générale de l'application est découpé en 3 programmes shell. GestionJeu, JoueurHumain, JoueurRobot.

\subsection{Gestion Jeu}
\subsubsection{L'initialisation}

Une première phase de l'initialisation définit le nombre de joueur. Elle le fait en demandant le nombre de joueur humain puis le nombre de joueur robot. On lance en conséquence les scripts nécessaire au travers d'un terminal xTerm. Chaque joueur humain obtient un terminal au travers du quel il peut interagir.
\newline
\newline
Une deuxième phase de l'initialisation s'occupe de chercher le nombre maximum de tour / carte maximum que l'on peut distribuer. Il s'agit d'un petit algorithme qui incrémente une variable en fonction d'une condition. Cette condition est la suivante : si la variable multiplier par le nombre de joueur est inférieur à 100, alors on incrémente. On continue jusqu'à que la multiplication soit supérieur à 100. On en déduit à la fin le nombre maximum de tour.
\newline
\newline
La troisième phase de l'initialisation appel la fonction qui gère les cartes. Nous allons décrire cette fonction dans la sous section suivante puisque elle est appelé à différent moment de la partie.  

\subsubsection{La gestion des cartes}

La gestion des cartes a trois objectifs :

\begin{itemize}
	\item Mélanger aléatoirement les cartes
	\item Distribuer aux joueurs les cartes de 0 à N ( N = indice du rond multiplier par le nombre de joueur )
	\item Trier les cartes distribuer pour savoir dans quel ordre elles doivent être jouer
\end{itemize}

Ces trois objectifs devant être atteint de manière récurrente dans le code, ils sont contenu dans une fonction. Cette fonction est appelé après chaque ronde, à chaque fois qu'une mauvaise carte est joué et lors de l'initialisation. 
\newline
\newline
Le troisième objectif est motivé par une idée simple : on souhaite juste savoir la carte qui doit être jouer sans savoir quel joueur doit la jouer. 

\subsubsection{La gestion du jeu}

Imaginons que nous avons une API (elle est détaillé dans la section communications). Cette API nous permet de savoir une chose importante : 
Une carte a été jouer et on connaît son nombre. C'est ici que tout prend son importance car nous allons effectuer en conséquences des traitements à partir de cette action.\newline
On regarde tout d'abord si la carte jouer est la carte qui devait être jouer (merci l'objectif 3 de la gestion de cartes). Si ce n'est pas le cas, on notifie les joueurs que la carte tirée est mauvaise et on redistribue des cartes (fonction gestion des cartes) et le tour recommence. Sinon, on notifie tout les joueurs de la carte trouver. On regarde ensuite si il s'agissait de la dernière carte du tour. Si c'est le cas et qu'il reste un tour, on passe au tour suivant en redistribuant des cartes. 

\subsection{Joueur Humain}
\subsubsection{Le comportement du script}
\subsubsection{Les problèmes de commandes bloquantes}
\subsubsection{La résolution}

\subsection{Joueur robot}
\subsubsection{Une copie de joueur humain ?}
\subsubsection{Sa stratégie}
Il a été difficile d'établir une stratégie pour le robot. Le problème a donc été décalée et on se repose sur le joueur. L'idée est simple. Quand le joueur estime que ce n'est pas son tour, il va laisser jouer. Ainsi, au bout d'un certains temps, le robot joue sa carte la plus faible. Malgré des faiblesses cette stratégie reste très forte car repose sur l'intelligence du joueur humain. 

\section{La communication}

Il existe dans ce projet plusieurs types de communications. On utilise les pipes pour communiquer des données et notifier des actions et les fichiers tmp pour synchroniser des données entre les shell et subshell

\subsection{Les pipes : une micro API}

Chaque terminal utilisent constamment des pipes pour recevoir ou envoyer des données / actions. Il a été choisit de construire une micro API pour facilité cette communication. On envoie un message dans les pipes constituer de la forme suivante :
\begin{center}
[n°action];[message]
\end{center}
Le ";" permet de facilité la séparation et récupération du n°action et du message. Voici les n°action que l'on peut trouver:
\begin{itemize}
	\item N°1 décrit que la carte trouvée était la bonne.\newline Le message sera du type : \textit{Bravo, une carte a été trouvée, voici les cartes trouvées : (1 2 )}
	\item N°2 décrit que la carte trouvée était mauvaise.\newline Le message sera du type : \textit{Perdu, la carte 2 n'était pas la bonne, la bonne était : 1. On recommence !}  
	\item N°3 décrit que le tour est terminé, on passe au suivant.\newline Le message sera du type : \textit{Félicitations, le tour n°'2 est terminé, on passe au tour suivant} 
	\item N°4 décrit que le jeu est terminé, on passe au suivant.\newline Le message sera du type : \textit{Félicitations, le jeu est terminé}
	\item N°5 décrit que le joueur a reçu toutes ses cartes dans son fichier tmp     
	\item N°9 est spécifique au robot. Il répond à un problème technique. Brièvement à chaque fois qu'une carte est jouer le robot s'envoie à lui même le message "9;distance" au bout d'un certains temps. "distance" étant la distance entre sa carte la plus faible et la dernière carte jouée. Comme ça si une nouvelle carte est jouée entre temps, la distance change et on laisse au joueur le temps de réévaluer la situation.  
\end{itemize}

\subsection{Les fichiers tmp}
Les fichiers tmp sont présent pour résoudre le problème de synchronisation des variables entre le shell et le subshell (problème dû aux commandes bloquantes). Chaque ligne du fichier tmp représente une carte. On lit toutes les lignes pour savoir les cartes du joueur. Ainsi, pour transmettre des cartes, gestion jeu écrit dans le fichier tmp et envoit un message à travers les pipes pour décrire quand il a finit. Quand le joueur joue une carte, on réécrit toutes les cartes ligne à ligne dans le fichier sans celle jouée.  
%----------------------------------------------------------------------------------------

\end{document}
