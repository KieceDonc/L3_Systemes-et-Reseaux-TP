%%%%%%%%%%%%%%%%%%%%%%%%%%%%%%%%%%%%%%%%%
% Lachaise Assignment
% LaTeX Template
% Version 1.0 (26/6/2018)
%
% This template originates from:
% http://www.LaTeXTemplates.com
%
% Authors:
% Marion Lachaise & François Févotte
% Vel (vel@LaTeXTemplates.com)
%
% License:
% CC BY-NC-SA 3.0 (http://creativecommons.org/licenses/by-nc-sa/3.0/)
% 
%%%%%%%%%%%%%%%%%%%%%%%%%%%%%%%%%%%%%%%%%

%----------------------------------------------------------------------------------------
%	PACKAGES AND OTHER DOCUMENT CONFIGURATIONS
%----------------------------------------------------------------------------------------

\documentclass{article}

\input{structure.tex} % Include the file specifying the document structure and custom commands

%----------------------------------------------------------------------------------------
%	ASSIGNMENT INFORMATION
%----------------------------------------------------------------------------------------

\title{Système et réseaux : rapport} % Title of the assignment
\author{Valentin VERSTRACTE \& Evan PETIT}

\date{L3 --- \today} % University, school and/or department name(s) and a date



%----------------------------------------------------------------------------------------

\begin{document}

\maketitle % Print the title


%----------------------------------------------------------------------------------------
%	Introduction
%----------------------------------------------------------------------------------------

\section{Structure générale de l'application}
La structure générale de l'application est découpé en 3 programmes shell. GestionJeu, JoueurHumain, JoueurRobot.

\subsection{Gestion Jeu}
\subsubsection{L'initialisation}

L'initialisation définit le nombre de joueur. Elle le fait en demandant d'abord le nombre de joueur humain puis ensuite le nombre de joueur robot. On lance ensuite les scripts nécessaire au travers d'un terminal xTerm. 

\subsubsection{La gestion des cartes }

\subsection{Joueur Humain}
\subsubsection{Le comportement du script}
\subsubsection{Les problèmes de commandes bloquantes}
\subsubsection{La résolution}

\subsection{Joueur robot}
\subsubsection{Une copie de joueur humain ?}
\subsubsection{Sa stratégie}

\section{La communication}

\subsection{Les pipes}
\subsection{Les fichiers tmp}

\section{La gestion du temps}
%----------------------------------------------------------------------------------------

\end{document}
